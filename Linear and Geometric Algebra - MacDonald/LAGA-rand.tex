\documentclass{article}
\usepackage{amsfonts}
\usepackage[fleqn]{mathtools}
\usepackage{amssymb}
\usepackage[a4paper, total={6.5in, 9in}]{geometry}
\DeclareMathSizes{10}{10}{9}{7}
\hyphenpenalty=10000

\newcommand{\p}{\partial}
\newcommand{\fpd}[2][]{\frac{\partial #1}{\partial #2}}
\newcommand{\spd}[2][]{\frac{\partial^2 #1}{\partial #2}}
\newcommand{\hsp}[1][5]{\hspace{0.#1 cm}}
\newcommand{\hcm}[1][1]{\hspace{#1 cm}}
\newcommand{\bra}[1][1.2]{\scalebox{#1}{$\boldsymbol{\langle}$}}
\newcommand{\ket}[1][1.2]{\scalebox{#1}{$\boldsymbol{\rangle}$}}
\newcommand{\hh}{\hbar}
\newcommand{\dint}[4][0]{\int_{#1}^{#2} #3 \ d#4}
\newcommand{\lt}{\left}
\newcommand{\rt}{\right}
\newcommand{\lcm}{\text{lcm}}
\newcommand{\imp}{\Rightarrow}
\newcommand{\rimplies}{\Longleftarrow}
\newcommand{\rimp}{\Leftarrow}
\newcommand{\siff}{\Leftrightarrow}
\newcommand{\st}{\ : \ }

\newcommand{\N}{\mathbb{N}}
\newcommand{\Z}{\mathbb{Z}}
\newcommand{\Q}{\mathbb{Q}}
\newcommand{\R}{\mathbb{R}}
\newcommand{\C}{\mathbb{C}}
\newcommand{\G}[1][n]{\mathcal{G}_#1}

\newcommand{\ch}[1]{\text{#1}}
\newcommand{\tdeg}{$^{\circ}$}
\newcommand{\mdeg}{^{\circ}}

\makeatletter
\newcommand*\bigcdot{\mathpalette\bigcdot@{.5}}
\newcommand*\bigcdot@[2]{\mathbin{\vcenter{\hbox{\scalebox{#2}{$\m@th#1\bullet$}}}}}
\makeatother
\newcommand{\inr}{{\mspace{2mu}\bigcdot\mspace{2mu}}}
\newcommand{\otr}{\raisebox{0.1ex}{\scalebox{0.7}{$\boldsymbol{\wedge}$}}}
\newcommand{\x}{{\times}}
\newcommand{\rev}{^\dag}
\newcommand{\nsubg}{\trianglelefteq}

\newcommand{\FLIPINDEX}[1]{{\scalebox{1}[-1]{$\scriptscriptstyle{#1}$}}}
\newcommand{\FLIPARG}[1]{{\scalebox{1}[-1]{$\mkern2mu#1$}}}


\newcommand{\dspt}{\displaystyle} 

\begin{document}
\begin{center}
LAGA random notes
\end{center}
\begin{flushleft}

Hi, Peter here to explain the joke.\\\hcm

Euler's formula (well, one of them...) says $e^{\theta i} = \cos(\theta) + i\sin(\theta)$. So $e^{\theta i}$ is just shorthand for a vector of sorts, spinning in a circle according to $\theta$. We add vectors by putting the head of one on the tail of the other, hence why graphing $\dspt e^{\theta i} + e^{\pi\theta i}$ looks like that.\\\hcm

Suppose while graphing $\dspt e^{\theta i} + e^{\pi\theta i}$, we find it ``covers its own tracks'', i.e. there exists $\theta_0,\ \theta_1$ such that\\\hcm 

 $(\theta_0 = \theta_1 \mod 2\pi)$ and $(\theta_0\pi = \theta_1\pi \mod 2\pi)$. \\\hcm

I.e. we just want to see if there is an initial condition $\theta_0$ which comes with a final point $\theta_1$ that puts the two vectors defined in each half of the function in the same ``configuration'' that it started with.\\\hcm

Letting $\phi = \theta/2$, this is the same as saying:\\\hcm

$(\phi_0/\pi = \phi_1/\pi \mod 1)$ and $(\phi_0 = \phi_1 \mod 1)$\\\hcm

The second condition says that $\phi_0$ and $\phi_1$ are rational ``with respect to each other":\\\hcm $(\phi_0 = \phi_1 \mod 1) \iff (\phi_0 - \phi_1 = 0 \mod 1) \iff \phi_0 - \phi_1$ is an integer.\\\hsp

The first condition says similar:\\\hcm
$(\phi_0 - \phi_1)/\pi$ is some integer $\dspt d \iff \pi = \frac{\phi_0 - \phi_1}{d}$\\\hcm

So if the graph ever began tracing over itself, this is equivalent to saying we've found a rational expression for $\pi$. 


\end{flushleft}
\end{document}