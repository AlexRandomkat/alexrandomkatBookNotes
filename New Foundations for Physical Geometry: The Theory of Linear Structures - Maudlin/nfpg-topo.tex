\documentclass{article}
\usepackage{amsfonts}
\usepackage[fleqn]{mathtools}
\usepackage{amssymb}
\usepackage[a4paper, total={6.5in, 9in}]{geometry}
\usepackage{xcolor}
\DeclareMathSizes{10}{10}{9}{7}
\hyphenpenalty=10000

\newcommand{\p}{\partial}
\newcommand{\fpd}[2][]{\frac{\partial #1}{\partial #2}}
\newcommand{\spd}[2][]{\frac{\partial^2 #1}{\partial #2}}
\newcommand{\hsp}[1][5]{\hspace{0.#1 cm}}
\newcommand{\hcm}[1][1]{\hspace{#1 cm}}
\newcommand{\bra}[1][1.2]{\scalebox{#1}{$\boldsymbol{\langle}$}}
\newcommand{\nl}[1][12]{\\[#1pt]}
\newcommand{\ket}[1][1.2]{\scalebox{#1}{$\boldsymbol{\rangle}$}}
\newcommand{\hh}{\hbar}
\newcommand{\dint}[4][0]{\int_{#1}^{#2} #3 \ d#4}
\newcommand{\lt}{\left}
\newcommand{\rt}{\right}
\newcommand{\lcm}{\text{lcm}}
\newcommand{\imp}{\ \Rightarrow\ }
\newcommand{\rimplies}{\Longleftarrow}
\newcommand{\rimp}{\ \Leftarrow\ }
\newcommand{\siff}{\ \Leftrightarrow\ }
\newcommand{\st}{\ : \ }
\newcommand{\defeq}{\vcentcolon=}

\newcommand{\N}{\mathbb{N}}
\newcommand{\Z}{\mathbb{Z}}
\newcommand{\Q}{\mathbb{Q}}
\newcommand{\R}{\mathbb{R}}
\newcommand{\C}{\mathbb{C}}
\newcommand{\G}[1][n]{\mathcal{G}_#1}

\newcommand{\ch}[1]{\text{#1}}
\newcommand {\chb}[1]{\textbf{#1}}
\newcommand{\tdeg}{$^{\circ}$}
\newcommand{\mdeg}{^{\circ}}

\makeatletter
\newcommand*\bigcdot{\mathpalette\bigcdot@{.5}}
\newcommand*\bigcdot@[2]{\mathbin{\vcenter{\hbox{\scalebox{#2}{$\m@th#1\bullet$}}}}}
\makeatother
\newcommand{\inr}{{\mspace{2mu}\bigcdot\mspace{2mu}}}
\newcommand{\otr}{\raisebox{0.1ex}{\scalebox{0.7}{$\boldsymbol{\wedge}$}}}
\newcommand{\x}{{\times}}
\newcommand{\rev}{^\dag}
\newcommand{\nsubg}{\trianglelefteq}

\newcommand{\FLIPINDEX}[1]{{\scalebox{1}[-1]{$\scriptscriptstyle{#1}$}}}
\newcommand{\FLIPARG}[1]{{\scalebox{1}[-1]{$\mkern2mu#1$}}}

\newcommand{\seg}{\!-\!\!\!\!\:-}
\newcommand{\lseg}{-\!\!\!\leftarrow\!\!}
\newcommand{\rseg}{\!\rightarrow\!\!\!-}
\newcommand{\adj}{\circ\!\!\!\:\seg\!\!\!\!\:\circ\,}
\newcommand{\ladj}{\circ\!\!\!\!\:\lseg\!\!\:\circ\,}
\newcommand{\radj}{\,\circ\!\!\!\:\rseg\!\!\!\!\:\circ}

\newcommand{\dspt}{\displaystyle} 

\newcommand{\ul}[1]{\underline{#1}}

\begin{document}
\begin{center}
Theory of Linear Structures Notes
\end{center}
\begin{flushleft}
\hangindent=1.1cm 
$p\seg q\hcm p \lseg q \hcm p\rseg q \hcm p \adj q \hcm p\ladj q \hcm p\radj q$\\
Note: my own additions (besides notation) are in {\color{purple} purple}. Proceed with caution.\nl[6]
A topology on $X$ is a set of open sets $T\subseteq \mathcal{P}(X)$ such that:\nl[6]
 - $X \in T$\\
 - $\emptyset \in T$\\
 - $T$ is closed under arbitrary unions\\
 - $T$ is closed under finite intersections\nl[6]
A closed set is a complement of an open set.\\
Neighborhoods of $p$ contain an open set containing $p$.\\
$p$ is a boundary point of a set $S$ means every open set containing $p$ intersects both $S$ and $S^C$.\\
A connected set is one not formed by the union of two disjoint non-empty open sets.\\
A Hausdorff ($T_2$) space is one where every pair of distinct points lie in distinct open sets.\nl[7]

\par\noindent\rule{\textwidth}{0.4pt}\nl[5] %axioms

A \chb{Point-Spliced Linear Structure} on a set $S$ is an ordered pair $\bra S, \Lambda \ket\ \ \Lambda \subseteq \mathcal{P}(S)$ satisfying axioms $\ch{LS}_1,\ \ch{LS}_2,\ \ch{LS}_3,\ \ch{LS}_4$\nl[6]
\hsp[2]\textbf{lines} are elements of $\Lambda$\nl[12]

$\ch{LS}_1$: $\forall\ \lambda\ \big(\ |\lambda|\geq 2\ \big)$\nl[6]\hsp[2]
$\chb{minimal lines}$ are $\chb{lines}$ of size 2 ($p\adj q$ denotes the minimal line connecting $p,q$, also acts as a binary relation which is true if such a minimal line exists) \\\hsp[2]
$\chb{segment}_\lambda(\mu) \defeq \mu \subseteq \lambda$ and $\mu \in \Lambda$.\\\hsp[2]
$\chb{point}_\lambda$ denotes an element of the $\chb{line}\ \lambda $\\\hsp[2]
(read: ``segments on lambda, points on lambda")\nl[6]
\hsp[2] $r\  \chb{between}_\lambda\ p,q \ \defeq \ p,q,r \in \lambda$ and $ \Big( \chb{segment}_\lambda(\mu)$ and $p,q \in \mu \imp r \in \mu \Big)$.\\
\hsp[2] Denote $p \seg r \seg q \ch{ on }\lambda \defeq r\ \chb{between}_\lambda\ p,q\ \ $ (equivalent to $q \seg r \seg p$ on $\lambda$).\\
\hsp[2] For nondistinct points, e.g. $r\seg r\seg s$, the relation is false.\nl[12]

$\ch{LS}_2$: Every \chb{line} $\lambda$ admits a linear order $>$ such that $\mu \in \Lambda \iff \mu$ is an interval of $>$\nl[5]

{\color{purple}
	${\ch{LS}_2}^*$: Equivalently, specify that for every \chb{line} $\lambda$:\\
	
	\hcm[2]For any three distinct points $p,q,r$, there is exactly one choice of $r$ for which $p\seg r \seg q$\\
	
	\hcm[2]For all $\chb{points}_\lambda\ p,q\ $the sets $PQ = \{r \in \lambda \st p\seg r \seg q\}$ and $\overline{PQ} = PQ\cup \{p,q\}$ are $\chb{segments}_\lambda$\\\hcm[2](provided $|PQ|\geq2$)\nl[5]
	
	\hcm This is enough to make the betweeness relation ``behave nicely", as we have eliminated  forks as candidates for lines by requiring the existence of at least one $r$ and removed loops by the uniqueness of $r$, enforcing a linear order. So we can show a proof of equivalence:\nl[5]
	
	\hcm For any $\chb{segment}_\lambda\ S$ containing $p, q$, we have $\overline{PQ} \subseteq S$. Supposing otherwise would mean there exists a $r \in \lambda \st p\seg r \seg q$ and $r \notin S$, which contradicts $S$ being a $\chb{segment}_\lambda\ S$ containing $p, q$ and the definition of $p\seg r \seg q$\nl[5]
	 
	\hcm Now, consider letting $p \seg r \seg s$ and $r\seg s\seg q$ on $\lambda$\nl[5]
	\hcm${\ch{LS}_2}^*$ says exactly one of $p\seg q \seg r$, $q \seg p \seg r$, or $p \seg r \seg q$ are true on $\lambda$. We have $p \seg q \seg r$ admits a $\chb{segment}_\lambda\ \overline{PR}$ containing $q$. But as $r\seg s \seg q$, $\overline{PR}$ must contain both $q$ and $s$. $\overline{PR}$ is the smallest $\chb{segment}_\lambda$ containing $p$ and $r$, so any $\chb{segment}_\lambda$ containing $p$ and $r$ also contains $s$, i.e. $p\seg s \seg r$, contrary to the assumption that $p \seg r \seg s$. A similar case holds for $q \seg p \seg r$, leaving $p \seg r \seg q$ as the only acceptable choice to satisfy ${\ch{LS}_2}^*$. Similarly, we also have $p \seg s \seg q$.\nl[5]
	
	\hcm So ${\ch{LS}_2}^*$ justifies the notation $p \seg r \seg s \seg q \defeq (p \seg r \seg s$ and $r \seg s \seg q)$, as all derivable true statements can be found by removing interior terms. Explicitly, this shows transitivity at both the interior and end points, so transitivity holds for all points, giving two symmetric linear orders. The trivial exceptions of lines with 2 or 3 points are easily handled. Both open and closed intervals (corresponding to $PQ$ and $\overline{PQ}$, respectively) are $\chb{segments}_\lambda$, hence ${\ch{LS}_2}^* \imp \ch{LS}_2$\nl[5]
}
	
\hcm The other direction $\ch{LS}_2 \imp {\ch{LS}_2}^*$ is easy: identify $p>r>q$ with $p\seg r\seg q$\nl[12]

\hcm {\color{purple}
	${\ch{LS}_2}^*$ is enough to push the $p\seg q$ notation further, allowing us to make well-formed longer chains. It would be easy to repeat the tricks above to show $p\seg x \seg r$ and $r \seg y \seg q \implies x \seg r \seg y$, which allows us to write $p\seg x \seg r \seg y \seg q$ as shorthand for all those true statements from deleting interior terms. But since we showed the existence of a linear order, there is no need for the now obvious.
}\nl[10]

\hsp[2] $\chb{endpoint}_\lambda(x) \defeq x \in \lambda$ and $\nexists\ p,q \in \lambda \st p\seg x \seg q$\\
\hsp[2] Denote $\chb{endpoints}_\lambda = \{p \st \chb{endpoint}_\lambda(p)\}$\\[6pt]\hsp[2]
$\lambda$ is \chb{open} $\ \ \, \defeq |\chb{endpoints}_\lambda|=0$\\\hsp[2]
$\lambda$ is \chb{closed} $\, \defeq |\chb{endpoints}_\lambda|=2$\\\hsp[2]
$\lambda$ is \chb{clopen}  $\defeq|\chb{endpoints}_\lambda|=1$\nl\hsp[2]
Theorem 2.1: A \chb{line} can have no more than 2 \chb{endpoints}.\\\hcm
{\color{purple}
	Alt Proof: Suppose the contrary. Consider ${\ch{LS}_2}^*$ and choose three endpoints. ${\ch{LS}_2}^*$ is immediately violated as there is no point between the other two.\nl[10]
}

$\ch{LS}_3$: If $\lambda \cap \mu$ contains only a common \chb{endpoint} $r$ and for all \chb{lines} $\gamma$ we have {\color{purple} $\gamma \subseteq (\lambda \cup \mu) - \{r\} \implies \gamma \subseteq \lambda$ or $\gamma \subseteq \mu$}, then $\lambda \cup \mu \in \Lambda$\nl[10]

\hsp[2] Theorem 2.2: Let \chb{lines} $\lambda$ and $\mu$ satisfy the conditions for point splicing with a common endpoint $r$. Let $p \in (\lambda/r),\ q\in (\mu/r)$. Then $p\seg r \seg q$ on $\lambda\cup \mu$.\\
\hcm {\color{purple}
	Alt proof: In contrapositive form, $\ch{LS}_3$: $ \gamma \nsubseteq \lambda$ and $\gamma \nsubseteq \mu \implies \gamma \nsubseteq (\lambda \cup \mu) - \{r\}$. So $p \in \gamma$ and $q\in \gamma \implies \gamma \nsubseteq (\lambda \cup \mu) - \{r\}$. If we take $\chb{segment}_{\lambda\cup\mu}(\gamma)$ (i.e. enforcing $\gamma\subseteq \lambda\cup\mu$), we must have $r\in \gamma$, i.e. $p\seg r\seg q$.\nl[10]
	
	\hcm (had random thought about the usage of proof by contradiction, compare this to Maudlin's proof. Which feels more like showing how the pieces fit exactly together in the right way? Like things gradually falling into place? It is better if every usage of proof by contradiction can be relegated to the automatic/unwritten level.)
}\nl[10]

$\ch{LS}_4$: Every set with a linear order $>$ such that the $\{\chb{closed lines}\} = \{\ch{closed intervals of }>\}$ is a line.\nl[5]

{\color{purple}
	${\ch{LS}_4}^*$:\nl[10]
}

\par\noindent\rule{\textwidth}{0.4pt}\nl[5] %completion of a linear structure

\chb{quasi-lines} are elements of a $\lambda$ which satisfies $\ch{LS}_1$, $\ch{LS}_2$, and $\ch{LS}_3$.\nl[2]
\hsp[2] a set $\sigma$ is \chb{closed-connected} $ := |\sigma| \geq 2 $ and $\exists$ a linear order $>$ such that \\\hcm$\{\chb{closed quasi-lines}$ in $\sigma\} = \{$closed intervals of $>\}$\nl[10]

\hcm Theorem 2.3: Given a Quasi-Linear Structure $\bra S, \Lambda\ket$, let $\Lambda^+$ denote the set of closed-connected subsets of $S$. Then $\bra S, \Lambda^+\ket $ is a Linear Structure.\nl[20]

A Proto-Linear Structure $\bra S, \Lambda\ket$ satisfies $\ch{LS}_1,\ \ch{LS}_2$. Proto-lines are elements of $\Lambda$.\nl[10]

\hsp[2] Proto-lines $\lambda$, $\mu$ are point-spliceable iff they have only a single endpoint $p$ in common, and for all proto-lines $\gamma$, $\gamma \subseteq (\lambda \cup \mu) - \{p\} \implies \gamma \subseteq \lambda$ or $\gamma \subseteq \mu$\nl[10]

\hsp[2] Given a proto-linear structure $\bra S, \Lambda_N\ket,\ \Lambda_{N+1}$ is the set $ \Lambda_N$ plus the unions of all pairs of point-spliceable proto-lines in $\bra S, \Lambda_N\ket$\nl[10]

\hsp[2] Each $\lambda$ in $\Lambda_{N+1}$ has a pair of \chb{associated linear orders}.\nl[10]

\hcm Theorem 2.4: $\bra S, \Lambda_N\ket$ is a proto-linear structure $\implies \bra S, \Lambda_{N+1}\ket$ is a proto-linear structure\nl[10]

\hcm Given a proto-linear structure $\bra S, \Lambda_0\ket$, let $\Lambda_\infty$ denote $ \bigcup_{i=0}^\infty\Lambda_i$\nl[10]

\hcm Theorem 2.5: If $\bra S, \Lambda_0\ket$ is a proto-linear structure, $\bra S, \Lambda_\infty\ket$ is a quasi-linear structure\nl[10]

Given any proto-linear structure $\bra S, \Lambda_0\ket$, $\bra S, \Lambda_\infty^+\ket$ is the linear structure \chb{generated} from $\Lambda_0$\nl[10]

\par\noindent\rule{\textwidth}{0.4pt}\nl[5] %types of linear orders

Linear order properties: \chb{dense, complete, discrete}\\
corresponds to \chb{discrete space}, \chb{continuum}, \chb{rational space} respectively, \chb{uniform space}\nl[10]

\par\noindent\rule{\textwidth}{0.4pt}\nl[5] %neighborhoods and adjacency

$\sigma$ is a \chb{neighborhood} of $p$ means $p \in \sigma$ and $\chb{endpoint}_\lambda(p) \implies \exists\  (\chb{segment}_\lambda = \mu \subseteq \sigma) \ \chb{endpoint}_\mu(p)$\nl[10]

\hcm Theorem 2.6: $X\supseteq \sigma \implies X$ is a \chb{neighborhood} of $p$\nl[5]

\hcm Theorem 2.7: \{\chb{neighborhoods} of $p$\} are closed under finite intersection\nl[10]

Two points $p$ and $q$ are \chb{adjacent} means $\{p, q\}$ is a \chb{minimal line}.\nl

\hcm Theorem 2.8: In a \chb{discrete} linear structure, $\sigma$ is a \chb{neighborhood} of $p$ $\iff \sigma$ contains $p$ and all points \chb{adjacent} to $p$.\nl

$\sigma$ is an \chb{open set} iff $\sigma$ is a \chb{neighborhood} of all of its members.\nl

\hcm Theorem 2.9: The \chb{open sets} on any linear Structure  $\bra S, \Lambda\ket$ are a topology on $S$.\nl

\hcm Theorem 2.10: In a \chb{discrete} linear structure, a set $\sigma$ is \chb{open} iff there is no \chb{minimal line} intersecting $\sigma$ and $\sigma^C$.\nl[3]
\hcm (so the smallest non-empty \chb{open sets} partitions the \chb{discrete} space)\nl[10]

\par\noindent\rule{\textwidth}{0.4pt}\nl[5] %DLS, basic terms

A topology is \chb{inherently directed} iff it cannot be generated by a Linear Structure, but can be by a Directed Linear structur.\nl[10]

A \chb{directed line} $\underline{\lambda}$ is a set of points $\lambda$ together with a linear order $>_{\underline{\lambda}}$ on the set. (so a \chb{line} but with a preference for one of its \chb{associated linear orders})\nl[10]

\hsp[2] Two \chb{directed lines} $\underline{\lambda},\, \underline{\mu}$ are \chb{codirectional} iff for some pair of points $p$ and $q$, $p >_{\underline{\lambda}}q $ and $ p >_{\underline{\mu}}q$\nl[5]
\hsp[2] Two \chb{directed lines} $\underline{\lambda},\, \underline{\mu}$ are \chb{antidirectoinal} iff for some pair of points $p$ and $q$, $p >_{\underline{\lambda}}q $ and $ q >_{\underline{\mu}}p$\nl[5]
\hsp[2] Two \chb{directed lines} $\underline{\lambda}$ and $\underline{\mu}$ \chb{agree} iff $\lambda$ and $\mu$ have at least two points in common and for every pair of points $\{p,q\}$ that they have in common, $p >_{\underline{\lambda}}q \iff p >_{\underline{\mu}}q$\nl[5]
\hsp[2] Two \chb{directed lines} $\underline{\lambda}$ and $\underline{\mu}$ are \chb{opposite} iff $\lambda$ and $\mu$ have at least two points in common and for every pair of points $\{p,q\}$ that they have in common, $p >_{\underline{\lambda}}q \iff q >_{\underline{\mu}}p$\nl[5]
\hsp[2] Two \chb{directed lines} $\underline{\lambda},\, \underline{\mu}$ are \chb{inverses} iff $\lambda = \mu$ and $\underline{\lambda}$ is \chb{opposite} to $\underline{\mu}$\nl[10]

$\chb{segment}_{\underline{\lambda}}$ denotes a \chb{directed line} $\underline{\mu}$ for which $\mu \subseteq \lambda$ and $\underline{\mu}$ \chb{agrees} with $\underline{\lambda}$\nl[5]
\hsp[2] $\chb{inverse segment}_{\underline{\lambda}}$ denotes the \chb{inverse} of some $\chb{segment}_{\underline{\lambda}}$\nl[10]

\hsp[2] $\chb{initial endpoint}_{\ul{\lambda}}(p) \iff \nexists\ q \in \lambda\hsp p >_{\ul{\lambda}} q$\nl[5]
\hsp[2] $\chb{final endpoint}_{\ul{\lambda}}(p) \iff \nexists\ q \in \lambda\hsp q >_{\ul{\lambda}} p$\nl[10]

\hsp[2] A \chb{directed line} $\ul{\mu}$ is an $\chb{initial segment}_{\ul{\lambda}}$ iff $\ul{\mu}$ is a $\chb{segment}_{\ul{\lambda}}$ and $\nexists q \in \lambda$ such that $p \in \mu \implies p >_{\ul{\lambda}} q$\nl[5]
\hsp[2] A \chb{directed line} $\ul{\mu}$ is a $\chb{final segment}_{\ul{\lambda}}$ iff $\ul{\mu}$ is a $\chb{segment}_{\ul{\lambda}}$ and $\nexists q \in \lambda$ such that $p \in \mu \implies q >_{\ul{\lambda}} p$\nl[15]

\par\noindent\rule{\textwidth}{0.4pt}\nl[8] %DLS, axioms

A \chb{Point-Spliced Directed Linear Structure} is an ordered pair $\bra S, \ul{\Lambda} \ket$ satisfying axioms $\ch{LS}_1,\ \ch{LS}_2,\ \ch{LS}_3,\ \ch{LS}_4$\nl[10]

$\ch{DLS}_1$: $|\chb{directed line}| > 2$\nl[15]

$\ch{DLS}_2$: For every \chb{directed line} $\ul{\lambda}$, $\ul{\mu}$ is a $\chb{segment}_{\ul{\lambda}} \iff \ul{\mu}$ is an interval of $>_{\ul{\lambda}}$.\\\hcm\  Otherwise, for a \chb{directed line} $\ul{\mu}^*$, $\mu^* \subseteq \lambda \iff \ul{\mu}^*$ is an $\chb{inverse segment}_{\ul{\lambda}}$\nl[15]

$\ch{DLS}_3$: If $\lambda \cap \mu$ contains only a single point $p$ for which $\chb{final endpoint}_{\ul{\lambda}}(p)$ and $\chb{initial endpoint}_{\ul{\mu}}(p)$ and for all \chb{directed lines} $\gamma$ we have {\color{purple} $\gamma \subseteq (\lambda \cup \mu) - \{p\} \implies \gamma \subseteq \lambda$ or $\gamma \subseteq \mu$}, then $\lambda \cup \mu$ with the linear order that agrees with $>_{\ul{\lambda}}$ and $>_{\ul{\mu}}$ is a \chb{directed line}.\nl[15]

$\ch{DLS}_4$: Every linearly ordered set of points $\ul{\sigma}$ such that all and only the \chb{closed codirectional directed lines} whose points lie in $\sigma$ are closed intervals of $>_{\ul{\sigma}}$ is a \chb{directed line}.

\par\noindent\rule{\textwidth}{0.4pt}\nl[8] %Neighborhoods, Open Sets, Topologies again

$\chb{outward neighborhood}_p(\sigma) \iff p \in \sigma $ and $\forall\ \ul{\lambda}$\\\hcm\  $\Big(\chb{initial endpoint}_{\ul{\lambda}}(p) \imp \exists\ \ul{\mu}\ \big(\chb{segment}_{\ul{\lambda}}(\mu)$ and $\chb{initial endpoint}_{\ul{\mu}}(p)\ \big)\Big)$\nl[5]

\hsp[2] $\chb{generalized neighborhood}_p(\sigma) \iff \chb{outward neighborhood}_p(\sigma)$ and $\chb{inward neighborhood}_p(\sigma)$\nl[10]

$q \chb{ outward adjacent } p \iff \exists\ \ul{\lambda}\ \Big(\lambda = \{p,q\}$ and $q>_{\ul{\lambda}}p\Big)$\nl[5]
\hcm Theorem 2.11: In a \chb{discrete} Directed Linear Structure, $\chb{outward neighborhood}_p(\sigma) \iff p \in \sigma $ and $p \radj x \imp x \in \sigma$\nl[10]

In a Directed Linear Structure, $\chb{outward open}(\sigma) \iff \forall\ p\in \sigma\ \chb{outward neighborhood}_p(\sigma) $\nl[5]
\hcm Theorem 2.12: The collection of $\chb{outward open sets}$ on a Directed Linear Structure $\bra S, \ul{\Lambda}\ket$ form a topology on $S$.\nl[5]
\hsp[2] The \chb{outward topology} is this topology.\nl[10]

A topology on a point set is \chb{geometrically interpretable} iff it is the \chb{outward topology} of some Point-Spliced Directed Linear Structure. Otherwise, it is \chb{geometrically uninterpretable}.\nl[10]

Given a topology $T$ on $S$ and $p\in S$, {\color{purple} $\sigma_p(T) = \{x\in S \st \sigma \in T \imp x \in \sigma\}$} (intersection of all open sets containing $p$)\nl[10]

\hcm Lemma 2.1: In a finite-point topological space, every $\sigma_p(T)$ is an open set of $T$.\nl[10]

Given a topology $T$ on $S$ and $\rho \subset S$, {\color{purple} $\rho^* = \{x \in S \st \exists\ r \in \rho \ (\ x \in \sigma_r(T)\ )\}$} (union of all $\sigma_r(T),\ r\in \rho$)\nl[5]

\hcm Lemma 2.1: In a finite-point topological space, every $\rho^*$ is an open set.\nl[10]

Theorem 2.13: In any \chb{discrete} Directed Linear Structure,\\\hcm $\chb{outward open}(\sigma) \iff \nexists\ p\radj q\ \Big(p \in \sigma $ and $q \in \sigma^\mathsf{c}\Big)$\nl[10]

A Directed Linear Structure $\bra S, \ul{\Lambda}\ket$ is \chb{complete} $\iff \forall\ p,q\in S\ \big(\ p\ladj q$ and $p \radj q\ \big)$.\nl[10]

\hcm Lemma 2.3: If $q \in \sigma_p(T)$ and $q \neq p$, then $p\radj q$ is in the DLS constructed from $\bra S, T\ket$.\nl[10]

\hcm Theorem 2.14 (the Finite-Point DLS/Topology Theorem): Let $\bra S, T_\ch{target}\ket$ be a topological space in which S is a finite set. There exists a Point-Spliced DLS (DLS$_\ch{constructed}$) on $S$ that generates $T_\ch{target}$.

\end{flushleft}
\end{document}